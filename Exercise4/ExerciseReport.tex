\documentclass[
    12pt,
    a4paper,
    oneside, 
    headinclude,footinclude,
    BCOR5mm,
]{scrartcl}
%----------------------------------------------------------------------------------------
\usepackage{graphicx}
\usepackage{hyperref}
\usepackage{booktabs}
\usepackage{multirow}
\usepackage{listings}
\usepackage{color}
 
\definecolor{codegreen}{rgb}{0,0.6,0}
\definecolor{codegray}{rgb}{0.5,0.5,0.5}
\definecolor{codepurple}{rgb}{0.58,0,0.82}
\definecolor{backcolour}{rgb}{0.95,0.95,0.92}
 
\lstdefinestyle{mystyle}{
    backgroundcolor=\color{backcolour},   
    commentstyle=\color{codegreen},
    keywordstyle=\color{magenta},
    numberstyle=\tiny\color{codegray},
    stringstyle=\color{codepurple},
    basicstyle=\footnotesize,
    breakatwhitespace=false,         
    breaklines=true,                 
    captionpos=b,                    
    keepspaces=true,                 
    numbers=left,                    
    numbersep=5pt,                  
    showspaces=false,                
    showstringspaces=false,
    showtabs=false,                  
    tabsize=2
}
\lstset{style=mystyle}
%----------------------------------------------------------------------------------------
\usepackage{sectsty}
\allsectionsfont{\rmfamily \mdseries \scshape}
%----------------------------------------------------------------------------------------

\hyphenation{Fortran hy-phen-ation}

%----------------------------------------------------------------------------------------
\newcommand{\cmd}[1]{\texttt{#1}}
\newcommand{\exercisequote}[1]{%
    {\quad\bfseries \small From the problem formulation:}%
    \vspace{-.5em}%
    \begin{quote}\itshape %
        #1 %
    \end{quote}%
}
%----------------------------------------------------------------------------------------

\begin{document}
\begin{centering}
    {\scshape \LARGE Exercise 4 \par}
    {\scshape Priority Queue Sort Experiment \par}
    {\itshape \small Solution by Eva Bertels. \par}
\end{centering}

\section*{Problem Description}
There are different ways to generate sorted output that shows different performance for input of varying format and size. 
This report focuses on the performance analysis of the library sorting function versus a priority queue implementation.

The performance of both approaches is measured for implementations in C/C++, JAVA and Python. 
The performance metric regarded is runtime, the indicator used is seconds taken by a stopwatch.

\section*{Input}
Input is generated in growing size: $10^{1}, 10^{2}, \ldots , 10^{6}$.

\begin{enumerate}
\item A list of random integers, range 100 and $10^5$
\item A list of random floating point numbers, range 100 and $10^5$
\item A list of sorted integers, range 100
\item A list of sorted floating point numbers, range 100
\item A list of random strings of length 3
\item A list of sortend strings of length 3
\end{enumerate}


\section*{Implementation}

\subsection*{C/C++}

To test the running time of the library sorting function versus a priority queue structure in C/C++ I made use of the code from the demonstration in the exercise class on 19. September 2017. 

It shows the example of a list of 10 random numbers chosen from the range 0 -- 99 (seed 0 for reproducability).
For the experiments input of different sizes and different values are used.

\begin{lstlisting}[language=C++, caption=Sorting in C++]
#include<iostream>
#include<vector>
#include<algorithm>
#include<stdlib.h>

int main(int argc, char **argv) {
  srand(0);
  std::vector<int> myVector;
  
  for(int x = 0; x < 10; x++) {
    myVector.push_back(rand() % 100);
  }
  std::sort(myVector.begin(), myVector.end());

  for(int x = 0; x < 10; x++){
    std::cout << myVector[x] << std::endl;
  }
  return 0;
}
\end{lstlisting}

\begin{lstlisting}[language=C++, caption=Priority Queue in C++]
#include<iostream>
#include<vector>
#include<queue>
#include<algorithm>
#include<stdlib.h>

int main(int argc, char **argv) {
  srand(0);
  std::priority_queue<int,std::vector<int>,std::greater<int> > myPQ;
  for (int x = 0; x < 10; ++x){
    myPQ.push(rand() % 100);
  }

  for (int x = 0; x < 10; ++x) {
    std::cout << myPQ.top() << std::endl;
    myPQ.pop();
  }
  return 0;
}
\end{lstlisting}


%\appendix

%\begin{figure}[ht!]
%    \centering
%    \includegraphics[width=\textwidth]{}
%    \caption{}
%    \label{fig:2:diff_other}
%\end{figure}


\section*{Output}

\begin{table}
\caption{Runtime results}
\begin{center}
\begin{tabular}{c|ccc|ccc}
\hline
\multirow{2}{*}{Input} & \multicolumn{3}{c}{\textbf{Sorting}} & \multicolumn{3}{c}{\textbf{Queue}} \\
 & C/C++ & JAVA & Python & C/C++ & JAVA & Python	\\
\hline
random Int & 1.1 & 3.2 & 2.3 & -- & 2 & 8 \\
\hline
\end{tabular}
\end{center}
\label{tab:output}
\end{table}


\end{document}